%%%%%%%%%%%%%%%%%%%%%%%%%%%%%%%%%%%%%%%%%%%%%%%%% 
\chapter{Estado da arte} \label{sec_classificacao_de_risco}
%%%%%%%%%%%%%%%%%%%%%%%%%%%%%%%%%%%%%%%%%%%%%%%%% 

\blue{Nesta seção é apresentado o estado da arte do aprendizado de máquina aplicado ao combate da pandemia. Aprendizagem de Máquina é um ramo da Inteligência Artificial onde sistemas computacionais podem aprender a partir de dados e identificar padrões com o mínimo de intervenção humana \cite{mitchell1997machine}. A Tabela \ref{tab_trabalhos_relacionados} apresenta o resumo de métodos de aprendizagem de máquina, métricas de avaliação e características aplicadas ao contexto da pandemia de COVID-19.
Os métodos identificados tem por objetivo auxiliar na tomada de decisão médica, incluindo prever a severidade dos casos confirmados (classificação de risco), chance de óbito ou recuperação, e chance de internação em UTI.}

\blue{Segundo os dados coletados, o método Floresta Aleatória é o mais frequentemente utilizado (aparece em 80\% dos casos).  Isto ocorre devivo ao fato deste método ser considerado acurado e robusto pois conta com um alto número de árvores de decisão no processo de treinamento, por não sofrer com sobreajuste e pelo seu auxílio na classificação da importância das características \cite{breiman2001random}. Devido às características mencionadas, utilizamos este algoritmo no processo de desenvolvimento do modelo preditivo para classificação de risco de mortalidade de pacientes com COVID-19.}

\begin{table}[!htb]  \footnotesize
\centering
\caption{Métodos de aprendizagem de máquina, métricas de avaliação e características}
\label{tab_trabalhos_relacionados}
\begin{tabular}{| p{4.0cm} | p{4.3cm} | p{2.5cm} | p{3.7cm} | }
\hline\hline
&
  \textbf{Métodos de aprendizagem de máquina usados} &
  \textbf{Características} &
  \textbf{Métricas de avaliação} \\ \hline

  
\textbf{\cite{iwendi2020covid}} & Floresta Aleatória* Intensificada (AdaBoost); Árvore de decisão; Máquina de Vetores de Suporte (SVM); Naive Bayes Gaussiano & Demográficas; Clínicas & 
Acurácia; Precisão; Revocação; F1 Score; Matriz de confusão \\ \hline
  
\textbf{\cite{pourhomayoun2020predicting}} &
  Redes Neurais Artificiais; Árvore de decisão; Floresta Aleatória*; K-vizinhos mais próximos (KNN); Regressão Logística; Máquina de Vetores de Suporte (SVM) &re
  Demográficas; Clínicas &
Acurácia; ROC-AUC;  Matriz de confusão \\ \hline
  
\textbf{\cite{zhao2020prediction}} &
  Regressão Logística &
  Clínicas &
  ROC-AUC \\ \hline
  
\textbf{\cite{yadaw2020clinical}} &
  eXtreme Gradient Boosting (XGBoost); Regressão Logística; Floresta Aleatória*; Máquina de Vetores de Suporte (SVM) &
  Demográficas; Clínicas &
  ROC-AUC \\ \hline
  
\textbf{\cite{cheng2020using}} &
  Floresta Aleatória* &
  Clínicas; Laboratoriais &
  Revocação; Especificidade; Acurácia; ROC-AUC; Matriz de confusão\\ \hline
  
\textbf{\cite{assaf2020utilization}} &
  Redes Neurais Artificiais; Floresta Aleatória*; Árvore de Decisão &
  Clínicas &
  Revocação; Especificidade; Acurácia; F1 Score; ROC-AUC;  Matriz de confusão\\ \hline
  
\textbf{\cite{gao2020machine}} &
  Regressão Logística; Máquina de Vetores de Suporte (SVM); eXtreme Gradient Boosting (XGBoost); Redes Neurais Artificiais; K-vizinhos mais próximos (KNN) &
  Clínicas; Laboratoriais &
  Razão de probabiliadde positiva; Razão de probabilidade negativa; F1 score; ROC-AUC; Matriz de confusão\\ \hline
  
\textbf{\cite{chowdhury2020early}} &
  eXtreme Gradient Boosting (XGBoost);  Regressão Logística &
  Demográficas; Clínicas; Laboratoriais &
  Revocação; Especificidade; Razão de probabiliadde positiva; Razão de probabilidade negativa; ROC-AUC; Matriz de confusão \\ \hline
  
\textbf{\cite{casiraghi2020explainable}} &
  Floresta Aleatória* &
  Radiológicas; Clínicas; Laboratoriais &
  Revocação; Especificidade; Acurácia; F1 score; ROC-AUC; Matriz de confusão \\ \hline
  
\textbf{\cite{dun2020machine}} &
  Floresta Aleatória* &
  Demográficas; Clínicas &
  -- \\ \hline
\end{tabular}
\end{table}

As principais características utilizadas para treinamento são dados demográficos (\textit{e.g.}, idade, sexo e país de origem), presentes em 50\% dos trabalhos; dados clínicos (\textit{e.g.}, sintomas e comorbidades), presentes em 100\% dos trabalhos; dados laboratoriais (\textit{e.g.}, amostras de sangue), presentes em 40\% trabalhos, onde foi possível identificar a mínima saturação de O2 como característica chave. 
Por fim, características radiológicas (\textit{e.g.}, imagens de radiografia) também aparecem como características de treinamento \cite{casiraghi2020explainable}.

Nos modelos propostos e treinados, a quantidade de características usadas varia de 3 \cite{yadaw2020clinical} a 99 \cite{cheng2020using}. 
Entretanto, foi possível observar que a idade e o sexo foram unânimes nos estudos que utilizaram de dados demográficos. 
Além disso, é importante ressaltar que nos modelos que utilizam dados clínicos, em termos de sintomas e comorbidades, todos utilizaram características relacionadas a resfriados, como febre, tosse e dispnéia.

Para entender quão bom um modelo preditivo é, existem métricas para avaliar o desempenho (ou generalização) de um método de aprendizagem de máquina \cite{amidi2020ml}. 
As métricas mais encontradas nos trabalhos foram métricas obtidas através da matriz de confusão, sendo a AUC-ROC a mais utilizada (presente em 80\% dos trabalhos), seguida por acurácia, precisão, sensibilidade, especificidade e F1 \textit{score} (presentes em 70\% dos trabalhos).

Os estudos mostram que modelos de aprendizagem de máquina conseguem prever chances de recuperação ou óbito com até 94\% de acurácia, com um F1 \textit{score} de 0.84 \cite{iwendi2020covid}. 
Essa métrica (F1 \textit{score}) auxilia na verificação do quão bom e generalista o modelo é, visto que é uma média harmônica entre a precisão (verdadeiros que realmente eram verdadeiros) e a sensibilidade (proporção dos verdadeiros positivos entre todas as observações que realmente são positivas). 
Isto demonstra a importância de outras métricas para avaliação dos modelos, além da acurácia, que é tipicamente a métrica mais utilizada em problemas de classificação binária.

Os trabalhos relacionados na Tabela \ref{tab_trabalhos_relacionados} possuem uma limitação em comum: o grupo de pacientes utilizados para treino. Um modelo de previsão aplicado em um novo sistema de saúde,
configuração ou país geralmente produz previsões que estão mal calibradas e podem precisar ser atualizadas antes que possam ser aplicadas com segurança nessa nova configuração. 
Os mesmos métodos, com outras populações ou de outros sistemas de saúde, podem levar a resultados diferentes  \cite{assaf2020utilization, cheng2020using, zhao2020prediction, casiraghi2020explainable}. 

Dados de vários países e sistemas de saúde podem permitir uma melhor compreensão da generalização e implementação de modelos de previsão em diferentes configurações e populações \cite{wynants2020prediction}. É com esta premissa que
neste trabalho serão utilizados dados abertos, as mesmas técnicas de aprendizagem de máquina, avaliação e validação do estado da arte, porém aplicadas à população do Rio Grande do Sul.


